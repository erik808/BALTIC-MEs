\documentclass{article}

\usepackage{xspace, natbib, minted}

\setminted{style=emacs, fontsize=\scriptsize, breaklines, frame=leftline}

\renewcommand{\deg}{$^{\circ} $ \xspace}
\newcommand{\ifile}[1]{\textit{#1.nc}\index{Input NetCDF files!#1.nc}}
\newcommand{\mdl}[1]{\textit{#1.F90}\index{Modules!#1}}
\newcommand{\ngn}[1]{\textit{#1}\index{Namelist Group Name!#1}}
\newcommand{\np}[1]{\textit{#1}\index{Namelist variables!#1}}

\begin{document}

DMP\_TOOLS can be used to generate a netcdf file containing the restoration coefficient $\gamma$.
Note that in order to maintain bit comparison with previous NEMO versions DMP\_TOOLS must be compiled and
run on the same machine as the NEMO model.
A \ifile{mesh\_mask} file for the model configuration is required as an input.
This can be generated by carrying out a short model run with the namelist parameter \np{nn\_msh} set to 1.
The namelist parameter \np{ln\_tradmp} will also need to be set to .false. for this to work.
The \ngn{nam\_dmp\_create} namelist in the DMP\_TOOLS directory is used to specify options for
the restoration coefficient.

%--------------------------------------------nam_dmp_create-------------------------------------------------
\inputminted{fortran}{./namelist}
%-------------------------------------------------------------------------------------------------------

\np{cp\_cfg}, \np{cp\_cpz}, \np{jp\_cfg} and \np{jperio} specify the model configuration being used and
should be the same as specified in \ngn{namcfg}.
The variable \np{lzoom} is used to specify that the damping is being used as in case \textit{a} above to
provide boundary conditions to a zoom configuration.
In the case of the arctic or antarctic zoom configurations this includes some specific treatment.
Otherwise damping is applied to the 6 grid points along the ocean boundaries.
The open boundaries are specified by the variables \np{lzoom\_n}, \np{lzoom\_e}, \np{lzoom\_s}, \np{lzoom\_w} in
the \ngn{nam\_zoom\_dmp} namelist.

The remaining switch namelist variables determine the spatial variation of the restoration coefficient in
non-zoom configurations.
\np{ln\_full\_field} specifies that newtonian damping should be applied to the whole model domain.
\np{ln\_med\_red\_seas} specifies grid specific restoration coefficients in the Mediterranean Sea for
the ORCA4, ORCA2 and ORCA05 configurations.
If \np{ln\_old\_31\_lev\_code} is set then the depth variation of the coeffients will be specified as
a function of the model number.
This option is included to allow backwards compatability of the ORCA2 reference configurations with
previous model versions.
\np{ln\_coast} specifies that the restoration coefficient should be reduced near to coastlines.
This option only has an effect if \np{ln\_full\_field} is true.
\np{ln\_zero\_top\_layer} specifies that the restoration coefficient should be zero in the surface layer.
Finally \np{ln\_custom} specifies that the custom module will be called.
This module is contained in the file \mdl{custom} and can be edited by users.
For example damping could be applied in a specific region.

The restoration coefficient can be set to zero in equatorial regions by
specifying a positive value of \np{nn\_hdmp}.
Equatorward of this latitude the restoration coefficient will be zero with a smooth transition to 
the full values of a 10\deg latitud band.
This is often used because of the short adjustment time scale in the equatorial region
\citep{Reverdin1991, Fujio1991, Marti_PhD92}.
The time scale associated with the damping depends on the depth as a hyperbolic tangent,
with \np{rn\_surf} as surface value, \np{rn\_bot} as bottom value and a transition depth of \np{rn\_dep}.

\bibliography{doc_DMP_tool}
\bibliographystyle{plainnat}

\end{document}
